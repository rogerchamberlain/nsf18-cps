\section{Broader Impacts}
\label{sec:broader}

The research will impact on the built environment.
According to the EPA, buildings are responsible for producing 6\% of
greenhouse gasses and heat and electrical generation produces another
25\%\footnote{\url{https://www.epa.gov/ghgemissions/global-greenhouse-gas-emissions-data}}.
A large portion of the electrical consumption in buildings is used for
heating, cooling and artificial lighting. Our proposal addresses the
reduction of electricity for artificial lighting to be replaced by
reflected daylight and capturing the solar heat. During daytime hours,
daylight is a preferred source of light for many people and the proposal
directs the daylight deeper into a building. The daylight reflection system
provides a more sustainable approach by reducing the required electricity
and provides a more desirable quality of light.

At the undergraduate education level, this work is closely related to
our second-semester introductory course in computer science and
computer engineering.  The
text (co-authored by PI R. Chamberlain) is entitled {\it Computing
in the Physical World}, and the course provides an introduction to
cyber-physical systems concepts in a laboratory-based setting.

The computational platform used in the course (an Arduino Uno) is the
same one used to control the Steinberg prototype catoptric surface,
and it has a very large hobbyist following (in the maker community).
We regularly request support for Research Experience for Undergraduate
(REU) students, and individuals who have completed
the above course will be well prepared to contribute to the research.

At the graduate education level, this work will support 4 graduate
students at Washington Univ.~in St.~Louis.
These students will be some combination of engineering students and
architecture students, with each community of students learning from
the other to broaden their individual horizons of experience to
include multidisciplinary work.

The Steinberg prototype was substantially designed by a student pursuing
dual degrees in architecture (MArch) and engineering (MEng)~\cite{Mitchell18}.
He recently was asked to present to the Washington University Board
of Trustees about his experience, and the university is considering
offering educational offerings that are tailored to students with
similar, cross-disciplinary interests.

We will leverage a pair of existing university programs to help us
attract students from traditionally underrepresented groups.  The Olin
Fellowship Program (for women) and the Chancellor's Fellowship Program
(aimed at underrepresented minority students) have had a successful track
record of enabling individuals to pursue graduate study.  In our
experience, the most effective method for attracting students from
underrepresented groups is by personal contact with a suitable role
model.  To facilitate this, we regularly ask the appropriately
qualified individuals in our group to be actively involved in the
recruiting process.  This cohort currently includes two
minority graduate students (one African-American student and one hispanic 
student).
We will attempt to leverage the maker space community as one target
for broadening participation from traditionally underrepresented groups.
