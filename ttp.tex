\documentclass[11pt]{article}
\usepackage{epsfig}
\usepackage{subfig}
\usepackage{setspace}
\usepackage{url}
\usepackage{colortbl} 

\setlength{\textheight}{8.875in} \setlength{\textwidth}{6.5in}
\setlength{\topmargin}{0.0in} \setlength{\headheight}{0.0in}
\setlength{\headsep}{0.0in} \setlength{\oddsidemargin}{0.0in}

%for comments
\newcommand{\FIXME}[1]{\textcolor{red}{FIXME: #1}}

\begin{document}
\pagestyle{plain}
\thispagestyle{plain}

\begin{center}
\textbf{\Large Transition to Practice}
\end{center}

In what follows, we describe the transition to practice option for the
research proposal.  We start with an articulation of the specific goals
of the transition to practice, and follow that with a discussion of the
execution plan.  We conclude with the budget estimate.

\section{TTP Goals}

Specific goals we wish to accomplish as part of our transition to practice
include the following.

\begin{enumerate}

\item \emph{Revise the design of the catoptric surface so that it can
be installed outside.}

Our initial two surfaces (already in place),
and those that we will construct as part of the research plan,
are all intended for indoor installation (e.g., in an atrium
or on or near a window.  This simplifies the design in a number of ways,
primarily enabling us to focus on the physical motion of the mirrors, without
having to ensure that each component is also weatherproof.

If the the system can be installed outside, that significantly increases
the circumstances under which it can be used practically.
\FIXME{Chandler, any chance you can expand upon this a bit? What words you have
might go here or in the execution plan below.}

\item \emph{Design and implement a user-friendly interface to the
open source software.}

As part of the research plan, we intend to design and implement two
levels of control software (one, low-level positioning control for a
pan-tilt physical system, and two, high-level management software for
Markov decision process optimization).  All of this software will be
released with an open source distribution.

However, during the research phase, the user interfaces to this software
will not be the primary focus of our effort.  As part of the transition
to practice, we will design and build an appropriate user interface that
enables the architecture community to both use each component individually
and to integrate them seamlessly.  This user interface capability will
also be released under an open source license.

\item \emph{Investigate the commercialization options for the system.}

We will explore two paths to assessing the potential for commercialization
of catoptric systems.
First, we will ask a number of practicing architects to evaluate the
catroptic surface and provide us with their professional opinions
(see letters).
Second, we will work with the manufacturing firm that is our second
prototype installation site, BECS Technology, to assess the commercial
viability from the point of view of a manufacturer (see letter).

Our group has significant experience in commercialization efforts.
PI R.~Chamberlain has formed 4 companies over the course of his professional
career, 3 of which are still in operation and collectively employ
over 200 individuals in St.~Louis, Chicago, New York, and London, UK.
Co-PI C.~Ahrens has \FIXME{Chandler, add some
bragging words here about your past efforts.}

\end{enumerate}

\section{TTP Plan}

The duration of the transition to practice plan will be for 3 years.
The first two years will overlap with the primary research plan (and
are labelled years~2 and~3 in what follows).  The third year will extend
beyond the primary research plan and is labelled year~4.

\FIXME{Add actual things to do in each year.}

\section{TTP Budget}

The execution of the TTP plan will extend the duration of the research
project for 1 year, and will involve the addition of one more individual
to the team.  Ed Richter, Professor of Practice in Electrical and Systems
Engineering, will join the team in years 2 through 4 and will have a
substantial role in executing the transition to practice activities.

\end{document}
