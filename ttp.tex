\documentclass[11pt]{article}
\usepackage{epsfig}
\usepackage{subfig}
\usepackage{setspace}
\usepackage{url}
\usepackage{colortbl} 

\setlength{\textheight}{8.875in} \setlength{\textwidth}{6.5in}
\setlength{\topmargin}{0.0in} \setlength{\headheight}{0.0in}
\setlength{\headsep}{0.0in} \setlength{\oddsidemargin}{0.0in}

%for comments
\newcommand{\FIXME}[1]{\textcolor{red}{FIXME: #1}}

\begin{document}
\pagestyle{plain}
\thispagestyle{plain}

\begin{center}
\textbf{\Large Transition to Practice}
\end{center}

In what follows, we describe the transition to practice option for the
research proposal.  We start with an articulation of the specific goals
of the transition to practice, and follow that with a discussion of the
execution plan.  We conclude with the budget estimate.

\section{TTP Goals}

Specific goals we wish to accomplish as part of our transition to practice
include the following.

\begin{enumerate}

\item \FIXME{item 1}

\item \FIXME{item 2}

\end{enumerate}

\section{TTP Plan}

\section{TTP Budget}

The execution of the TTP plan will extend the duration of the research
project for 1 year, and will involve the addition of one more individual
to the team.  Ed Richter, Professor of Practice in Electrical and Systems
Engineering, will join the team in years 2 through 4 and will have a
substantial role in executing the transition to practice activities.

\end{document}
