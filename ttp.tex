\documentclass[11pt]{article}
\usepackage{epsfig}
\usepackage{subfig}
\usepackage{setspace}
\usepackage{url}
\usepackage{colortbl} 

\setlength{\textheight}{8.875in} \setlength{\textwidth}{6.5in}
\setlength{\topmargin}{0.0in} \setlength{\headheight}{0.0in}
\setlength{\headsep}{0.0in} \setlength{\oddsidemargin}{0.0in}

%for comments
\newcommand{\FIXME}[1]{\textcolor{red}{FIXME: #1}}

\begin{document}
\pagestyle{plain}
\thispagestyle{plain}

\begin{center}
\textbf{\Large Transition to Practice}
\end{center}

In what follows, we describe the transition to practice option for the
research proposal.  We start with an articulation of the specific goals
of the transition to practice, and follow that with a discussion of the
execution plan.  We conclude with the budget estimate.

\section{TTP Goals}

Specific goals we wish to accomplish as part of our transition to practice
include the following.

\begin{enumerate}

\item \emph{Revise the design of the catoptric surface so that it can
be installed outside.}

Our initial two surfaces (already in place),
and those that we will construct as part of the research plan,
are all intended for indoor installation (e.g., in an atrium
or on or near a window.  This simplifies the design in a number of ways,
primarily enabling us to focus on the physical motion of the mirrors, without
having to ensure that each component is also weatherproof.

If the the system can be installed outside, that significantly increases
the circumstances under which it can be used practically.
When the mirror units can be located outside, then they are able to
be exposed to daylight beyond any obstructions such as roof overhangs.
An overhang can block some of the high altitude angle sunlight during
summer months or in locations near the equator.
An additional benefit is that the mirror units can provide some
shading for the interior area just inside the window.
This could benefit interior spaces in two ways.
First, the solar heat gain on the floor or desks near a window
may be unwanted and the mirror units can direct the daylight to
where it is more useful.
Second, direct daylight can cause too much luminous intensity or
glare in the area around the window and our systems can redirect
the light to bounce off the ceiling to diffuse the light before
reaching the task height of the occupants. 

\item \emph{Design and implement a user-friendly interface to the
open source software.}

As part of the research plan, we intend to design and implement two
levels of control software (one, low-level positioning control for a
pan-tilt physical system, and two, high-level management software for
Markov decision process optimization).  All of this software will be
released with an open source distribution.

However, during the research phase, the user interfaces to this software
will not be the primary focus of our effort.  As part of the transition
to practice, we will design and build an appropriate user interface that
enables the architecture community to both use each component individually
and to integrate them seamlessly.  This user interface capability will
also be released under an open source license.

\item \emph{Investigate the commercialization options for the system.}

We will explore two paths to assessing the potential for commercialization
of catoptric systems.
First, we will ask a number of practicing architects to evaluate the
catroptic surface and provide us with their professional opinions
(see letters).
Second, we will work with the manufacturing firm that is our second
prototype installation site, BECS Technology, to assess the commercial
viability from the point of view of a manufacturer (see letter).

Our group has significant experience in commercialization efforts.
PI R.~Chamberlain has formed 4 companies over the course of his professional
career, 3 of which are still in operation and collectively employ
over 200 individuals in St.~Louis, Chicago, New York, and London, UK.
Co-PI C.~Ahrens is a licensed architect with 22 years of professional
experience, working on institutional, commercial, and residential
buildings that focus on sustainable fa\c{c}ade and mechanical systems.
The projects have been located around the world in Europe, the Middle East,
China and the United States. As an example, he was the main designer for
the engineering building for Cooper Union, which was the first
LEED Platinum rated laboratory building in the United States.
Many of his completed buildings have won numerous awards for their
sustainable technologies.
He is currently the owner of an architecture firm and continues
to employ sustainable technologies and techniques in his practice.  
Co-PI C. Gill has worked extensively with industry partners throughout his
research career, including developing software in collaboration with Boeing, 
BBN Technologies, and Honeywell that was used in manned flight demonstrations.

\end{enumerate}

\section{TTP Plan}

The duration of the transition to practice plan will be for 3 years.
The first two years will overlap with the primary research plan (and
are labeled years~2 and~3 in what follows).  The third year will extend
beyond the primary research plan and is labeled year~4.

\subsection{Year 2}

\begin{itemize}

\item Design, prototype, and test several mirror elements aimed at
outdoor installation. This will include protection of the metallic components,
electronic components, and stepper motors.  The assessment will include
both the degree to which they can withstand the elements and whether or
not there is any degredation in operational peformance (e.g., reduced
positioning accuracy or other limitation).

\item Perform initial user interface design for individual components.
\FIXME{Describe.}

\end{itemize}

\subsection{Year 3}

\begin{itemize}

\item Assess manufacturability of physical system.  In collaboration with
the design engineers at BECS Technology, 
we will evaluate the ability and cost of manufacturing the catoptric
surfaces in moderate to large volumes.

\item Prototype component user interfaces. This will include the individual
interfaces for the low-level positioning controllers 
and for the high-level optimization specification.

\end{itemize}

\subsection{Year 4}

\begin{itemize}

\item Integrate unified user interface. \FIXME{Describe.}

\item Assess commercial viability. \FIXME{Interacting with architects.}

\item Seek commercial partners. \FIXME{Describe.}

\end{itemize}

\section{TTP Budget}

The execution of the TTP plan will extend the duration of the research
project for 1 year, and will involve the addition of one more individual
to the team.  Ed Richter, Professor of Practice in Electrical and Systems
Engineering, will join the team in years 2 through 4 and will have a
substantial role in executing the transition to practice activities.

\end{document}
