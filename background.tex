\section{Background and Related Work}
\label{sec:background}

\FIXME{Describe first two installations.}

\FIXME{Literature review~\cite{azaise13,bwkk15,gb16,kt16,Leslie03}.}

Markov Decision Processes (MDPs)~\cite{puterman} represent a general approach
to modeling optimization problems and have been applied in a diverse set of
application areas~\cite{White93}. Examples include robotics~\cite{ab10}, 
economics~\cite{bs98}, experiment design~\cite{kb85},
medical decisions~\cite{ahsr10}, manufacturing~\cite{yyl04},
agriculture~\cite{Kristensen03},
and our own group's use in scheduling~\cite{gtsg08,tggs10}
and wireless spectrum management~\cite{mgc16}.

Our prior research has used Markov decision process
models~\cite{gtsg08} to generate resource management policies
off-line~\cite{gtgs09} for non-preemptive sharing of a resource
between multiple purposes at once on-line.  For example, a meter-tall robot's
camera (oriented by a pan-tilt unit similar to the ones we propose to
use in our multi-mirror catoptric installations) may be directed
downward to identify wire-frame chairs and other obstacles to
navigation that other sensors on the robot may have difficulty
detecting, or it may be directed upward to identify faces of people at
a reception whose images it can then capture. Given distributions of
the durations of intervals during which the camera would need to
remain pointed in a given direction to complete an individual task,
standard policy iteration techniques then can be used to generate
run-time policies that in expectation maximize an objective such as
adherence to a strictly proportional allocation of the resource over
time~\cite{gtsg08}, or even a more general definition of the utility 
of completing the different tasks at particular times~\cite{tggs10}.
We also showed that when different distributions of task completion
intervals can occur in different modes (e.g., when a robot moves
from room to room), it is possible to learn on-line what mode
the system is in, or if the mode is known what the distributions are,
but not both~\cite{gtgsuai10}.

However, policy iteration is exponentially expensive, and even the memory 
requirements to store complete policies for on-line use may be prohibitive 
in resource-limited systems.  We therefore focused next on the policies 
that were being generated from the models, and discovered consistent 
structure in those policies that allowed a reasonable heuristic 
approximation.  For simple proportional sharing, a single geometric partition
of a simplex could be calibrated to encode the appropriate policy accurately~\cite{gtspmgs10}.
For utility-based resource sharing multiple disjoint heuristics were needed but
the most effective one to use was clearly defined by problem parameters~\cite{tblwgs11}.

As a further illustration both of the applicability of MDP-based
policy iteration to generate effective resource management policies,
we applied similar techniques to manage a much different resource:
the transmission spectrum in wireless networks~\cite{mskgct13}.  Although 
the semantics of that resource differed radically from the pan-tilt camera, 
the MDP models were reasonably similar.  We extended the basic model to
include modulation as well as admission decisions, discovered and characterized
common structure among the policies that were generated, and again obtained
efficient and effective heuristic policies for on-line use~\cite{mgc16}.

In this proposal we adopt the definition used by Glaubius et al.~\cite{gtsg08}
of a (discrete-time) Markov decision process as a 5-tuple
$(\mathcal{X}, \mathcal{A}, T, R, \gamma)$, with \emph{states} designated
as $\chi \in \mathcal{X}$, \emph{actions} designated as $a \in \mathcal{A}$,
and a transition system, $T$, which gives the probability
$P_T (\chi' \mid \chi, a)$ of transitioning from state $\chi$ to
state $\chi'$ on action $a$.
The reward function $R(\chi, a, \chi') \in \mathbb R_{\ge 0}$ describes the
reward that accrues when transitioning from state $\chi$ to
state $\chi'$ via action $a$, under a discount factor, $\gamma$,
to ensure convergence of the long term reward.

\FIXME{Add transition into the next section.}
