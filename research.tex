\section{Research Description}
\label{sec:research}

Our proposed research will address each of the specific questions posed in 
the introduction, as follows.

\subsection{Question 1 -- What are the qualitative and quantitative benefits
that can be achieved for building daylighting and thermal management
through the use of catoptric systems?}

The benefits of natural light in human-occupied spaces are well
documented both in the research literature and in the popular
press~\cite{hhm15,Leslie03,ll01,Libby03,pce00,vandenW14},
and yet thermal effects also may be significant~\cite{bmbc13} 
(e.g., too much sunlight can increase temperature to unacceptable levels).
While many studies focus on the benefits of homogeneous light levels within a space~\cite{azaise13,bwkk15,gb16}, appropriate light levels may differ based on user needs.
The increased range of lighting effects offered by the independent positioning of mirrors
can provide functional benefits for people who need more or less
light, e.g., due to the age of the occupant or the task she or he is performing. For 
example, someone using a computer may want less light, since the screen is a source of
artificial light and glare from outside may make it more difficult to see, while 
someone reading from unlit physical media like a book may require higher light levels. 
Similarly, light levels required for vision may vary with the ages of the occupants: as 
people get older, they tend to need higher light levels and increased luminous contrast. 
Furthermore, regardless of age, different people tend to have have different visual 
acuity. Thus, developing a system that can provide a wide range of lighting effects 
using daylight rather than artificial light to accommodate these varying requirements is beneficial.

For thermal management, the literature on harvesting sunlight for
space heating is substantial (e.g.,~\cite{deW75,Hunt79,kbd76,Lunde80,smf08,wo06}, 
and we will simply leverage such well-understood techniques. As many sunlight-based
heating systems already use catoptric mechanisms (though primarily for sun
tracking), our primary interest and contributions here will be to
integrate a sunlight harvesting system to investigate and demonstrate
dual-purpose use of catoptrics in a single cyber-physical system, both
for illumination and for heating.

In addition to investigating how (1) natural light can be redirected to achieve
different lighting effects using programmable dynamic orientation of mirrors on 
pan-tilt units, and (2) to what extent those lighting effects can achieve diverse
objectives such as energy collection, temperature control, and people's quality 
of experience in a space, we will also explore (3) how Markov decision processes 
(MDPs) can be used to generate policies automatically for multi-objective control of 
catoptric systems.  The choice of MDPs to generate such policies is motivated in part 
by the influence of stochastic factors such as wireless network packet loss between
a high-level controller and the units that control the mirrors themselves, and 
environmental effects such as transient cloud cover on the natural light that is 
available.

Another motivation for using MDPs is their efficacy in generating highly regular 
policies that can be implemented efficiently at run-time.  For example, policies for
proportional sharing of a non-preemptive resource~\cite{gtgs09,gtsg08} can be 
approximated efficiently on-line with high accuracy~\cite{gtspmgs10}.  Scenarios
that are relevant to this research, such as using multiple catoptric arrays to
provide even and consistent lighting of a collaborative work area, or to generate
in inhomogeneous pattern according to an input image, have similarly
regular structure that an MDP-based approach can exploit.  Our prior work also
shows that custom policies can be generated~\cite{tblwgs11,tggs10} that account 
for various nuances (e.g., in this proposed research, individual variations in 
visual acuity and thus lighting needs, as well as constraints to keep thermal 
effects within desired ranges).

We will start with a single objective, matching a provided image map that represents a 
desired illumination pattern, and then generalize this to include the separate goal of 
thermal energy harvesting while keeping room temperatures within a specified range.
To illustrate what we mean by an image map, Figure~\ref{fig:maps}
shows a set of arbitrary example images, associated with the mirror
positions that will create those images.  The forward problem is determining
the resulting image given the mirror positions.  The optimization problem
is determining the required mirror positions given the desired image.

\begin{figure}[ht]
\centering
\includegraphics[width=0.9\linewidth]{figures/maps}
\caption{Example image maps (bottom) and associated mirror positioning (top).}
\label{fig:maps}
\end{figure}

For the purpose of designing an MDP, we will leverage the ability of the low-level
control mechanism to position each individual mirror as desired.  The MDP state-space 
will model higher-level management issues, e.g., which mirror should be pointed in which 
direction at which time, using distributions of mirror positioning latencies (based on 
the previous positions of the mirrors, network latency including delays due to wireless
network packet losses, etc.) and variations in natural light that we will capture 
through profiling studies.  We also will further refine these models, which intially 
will be based on simply commanding the pan-tilt units to point to specific 
\emph{locations} in space and allowing the lower-level controllers to enact those 
commands separately, to a more complete cyber-physical model in which positioning 
commands may include specific \emph{trajectories} for each pan-tilt unit including 
its initial and final position, and the ranges of accelerations, velocities, and/or 
orientations it should remain within during its trajectory along that path - this 
cyber-physical refinement is important both in its ability to enact more rapid 
adaptations to environmental variations and to perform dynamic lighting effects 
(e.g., to get the attention of the people in a space if an emergency alert were 
issued), and also in terms of tracking the wear-and-tear on the pan-tilt units and 
how that affects longevity and reliability of the catoptric arrays.

With a low-level controller in place, the state space, $\mathcal{X}$,
can encapsulate the set of mirrors pointed at each position in
the image map.  The set of actions, $\mathcal{A}$, will embody the movement
of one or more mirrors from their present position to a new position,
and the transition system, $T$, will encode the probabilistic variations
present in the available sunlight.  In the simplest case, this allows the 
reward function, $R$, to be a quantified match between the desired image 
map and the resulting daylight that is directed to each position in the 
physical space (e.g., using existing image comparison quantification 
techniques~\cite{ds99,my09,wbo97}).
For more complex cases involving different individualized lighting requirements,
multiple objectives, etc., we will \emph{convolve} the different objective
functions to achieve a single common reward function, as we have done
in our prior work~\cite{tblwgs11,tggs10}.

After an initial exploration of the MDP formulated as described above,
we will then proceed to expand the framework to include the additional
objective of harvesting heat.  We will quantify the benefits of each
objective using general utility functions~\cite{tggs10} in a manner previously
explored by our group. One candidate set of utility functions we will
investigate will be to prioritize daylighting performance, and only
allocate \emph{excess} sunlight to the HVAC subsystem. A second alternative
will be to proportionally provide sunlight to each goal up to the point
that the illumination can no longer benefit, at which point all additionally
available sunlight would go into heating.

In each of the above MDPs under investigation, we will use policy iteration
to capture a \emph{value-optimal} solution (which maximizes accrued value in 
expectation and is guaranteed to exist within Markov decision process
theory); however, this solution typically must be computed off-line, since
it is, in general, exponentially expensive to compute.  In parallel, we
will also explore the space of value-optimal solutions and seek to formulate
a computationally inexpensive heuristic that closely approximates the
true value-optimal solution.

\subsection{Question 2 -- How do we provide for the safety, reliability,
maintainability, and continued efficacy of these systems?}

The ability to control mirror positions over time is only useful if the
additional considerations of safety, reliability, etc., are managed
properly for the system as a whole. To address these issues, we will
investigate the degree to which they can be handled within the context
of each low-level controller versus being dealt with by
higher-level system-wide control. We anticipate some of these
considerations being incorporated as constraints within the
optimization process and others as additional goals that we wish to
maximize during the multi-objective optimization itself.  Let us first
consider those that will be addressed as constraints (e.g., safety).

\paragraph{Constraints.}
A commonly used safety constraint on a positioning subsystem is to
specify hard limits on the range of motion (e.g., of each mirror, individually).
This kind of limit can be imposed by the physical design of the pan-tilt
mechanics, simply by adding physical stops at the limit positions.

The more interesting constraint systems occur when the safety considerations
are no longer locally determinable, but are dependent upon context.
An example that is relevant for our catoptric surfaces is a limit on
the total light intensity that can be supported at various positions in the
field of view of the mirrors. When providing heat into the HVAC system,
we desire a high light concentration delivered to the heat transfer point.
When illuminating a physical space for human occupants, levels of light 
concentration above a certain threshold are not only undesirable, they may 
be patently unsafe.

These context-sensitive constraints add two additional requirements for
our proposed research approach.
First, the higher-level management system must be responsible for addressing them.
In our MDP formulation, we may either adjust the feasible state space, $\mathcal{X}$,
so that they are unreachable, or adjust the reward function to give sufficiently 
negative rewards so as to preclude the policy violating them~\cite{tblwgs11,tggs10}).
Second, we must also ensure that whatever choices are made by the optimization
system (whether it be via MDPs or some other approach), the actual
mirror positions are still constrained so as to not result in an
unsafe condition.  This will require feedback of some form on the actual
mirror positions and the ability to determine whether or not the actual
positions differ from the controlled positions.  This may be accomplished
either locally (e.g., via shaft encoders on the pan-tilt mechanism) or
globally (e.g., via a visual monitoring system and appropriate image analysis
software): we will explore, and compare the consequences of, both options.

\paragraph{Goals.}
Many of the additional considerations are more appropriately addressed as
additional (potentially competing) goals, adding to specific needs for
daylighting and heating.  An example here would be the
impact on reliability of the system when individual mirrors are moved.  As
with all electro-mechanical systems, the pan-tilt mechanism has a limited
lifetime, which can be significantly impacted by its usage duty cycle
(i.e., the more it is moved, and the more intense the accelerations and decelerations
it experiences, the sooner it will likely fail).

These are precisely the kinds of circumstances in which Markov decision
processes are useful. For the initial goals (of question~1), is is quite
likely that there are multiple optimal solutions (e.g., simply exchanging
where two mirrors point, with the goals unchanged).  However,
once we add in considerations of reliability, which are impacted by
frequency of use and the trajectories used to reposition the mirrors, 
we now have a much richer search space, whose complexity can be encoded
effectively in terms of probabilitic rewards.
The MDP-based optimization approach is particularly well suited for
this type of problem, and can find value-optimal solutions that incorporate
wear-and-tear-minimizing movement of mirrors between configurations, thereby 
diminishing maintainence costs and downtime during the normal use of the system.

The research task here is to develop, experiment with, and evaluate
Markov decision processes that simultaneously deliver daylight where it
is most beneficial, maintain the safety of the overall space,
and provide for the reliability, maintainability, and efficacy of the
catoptric surface itself.  For example, when we introduce accelerations
of the pan-tilt units into the MDP model as described above, we will bias
the reward function to avoid hard stops and rapid accelerations which both
can wear out the pan-tilt units and also may contribute to uncertainty in
positioning (similar to how wheel-slip must be addressed in mobile
robotics~\cite{mn87}).

\subsection{Question 3 -- Can we design abstractions that encapsulate
subsystems for effective reuse?}

Layered \emph{system architectures}\footnote{We use the term
\emph{system architecture}
to denote computer hardware/software architecture, distinguishing it from
architecture for the built world around us.}
have proven effective in providing standardized interfaces to
support portability and reuse of hardware and software,
while allowing new innovations and abstractions
to enrich system capabilities.
Even when many actual systems don't adhere strictly to official standards
(such as the OSI~\cite{osi} networking model or the POSIX~\cite{posix} operating
systems interface model), system designers, developers,
and users still benefit from the overall structure those models provide.

This occurs largely because standardized interfaces establish clear
boundaries of responsibility on which other layers may rely, while 
allowing an essential ``permission to tinker'' with various 
implementations between those boundaries. For cyber-physical systems, 
whose semantics include timing and physical properties not considered 
in previous cyber-only system architectures, there is significantly 
less experience with what interfaces, abstractions, and even broad 
system layers are truly common (and so perhaps could be standardized) 
versus the question of which other aspects are more likely to diverge 
between systems, and so should be free to do so~\cite{cag18}.
In this research, we will investigate the viability and utility of
two candidate abstractions: direct pan-tilt control of individual mirrors 
and MDP-based system-wide control.

\paragraph{Pan-tilt control.}
An initial version of a library (and associated programming interface) 
for pan-tilt control of our mirror units is already in place~\cite{Mitchell18}.
It is based upon an Arduino Uno microcontroller commanding the stepper
motors controlling the mirrors (up to 32 mirrors per Arduino),
with communication to the Arduino via a USB link.  Python software 
executing on a Raspberry Pi coordinates a set of up to 10~Arduinos using a USB hub.

While the initial implementation simply commands position, it is open-loop
and therefore requires periodic resetting (which we accomplish by moving each
pan-tilt unit to its stops in each dimension), and only one motor 
connected to each Arduino can be in motion at a time.
As part of the research project, we will extend this positioning control
software in the following ways:
\begin{enumerate}

\item support closed-loop control, either through the use of shaft encoding
sensors or image-based feedback or both; and

\item support path direction (including acceleration profiles) in addition
to positioning commands.

\end{enumerate}

We will also extend our current prototype for mirror control to include concurrent 
positioning of multiple mirrors at once -- although positioning a mirror takes only 
seconds currently, which is appropriate to adjust gradually to the progression of 
the sun's position throughout the day, for more dynamic lighting effects the latency 
to adjust an entire row of mirrors in an array may be noticable, and may introduce 
undesirable lighting artifacts at least temporarily as mirrors are adjusted.

We also will investigate semi-specular reflective surfaces or a
combination of specular mirror surfaces with lenses to diffuse the
daylight to reduce the contrast with the surrounding surface that does
not have light reflected onto it. We will perform light meter tests to
measure the lux and the visual impact of direct versus diffuse
daylight reflections.

\paragraph{MDP control.}
As is the case with pan-tilt control, we already have an initial functioning
piece of software that, given an MDP model, provides both the
value-optimal decision at any point in the design space as well as
the expected value of reward~\cite{mskgct13,tggs10}.
In the current edition, the user must provide the details of the MDP model
by authoring a set of software routines (in C++) that effectively build
an in-memory representation of the MDP model.

We will explore alternative interfaces, both textual and graphical,
for articulating the details of an MDP model.  This is particularly
challenging as the models themselves are frequently quite substantial
in size.

The MDP models themselves also will be extended to consider concurrent
positioning of mirrors, in part to ameliorate wear-and-tear on the
pan-tilt units, and in part to achieve new dynamic lighting effects
that can include combinations of sequential and concurrent
repositioning.

Both of these software systems (and their associated interfaces), constitute
abstractions that have strong potential for reuse in other contexts.
Pan-tilt positioning control is ubiquitious in imaging for use
with cameras (although we have a few features, such as concurrent movement
at scale, that are likely unique relative to previous systems),
and we have already used the initial versions of the MDP
software in multiple contexts~\cite{mskgct13,tggs10}.
