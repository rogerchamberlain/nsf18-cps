\section{Research Description}
\label{sec:research}

We will describe the research in terms of our approach to addressing each
of the specific questions posed in the introduction.

\subsection{Question 1 -- What are the qualitative and quantitative benefits
that can be achieved for building daylighting and thermal management
through the use of catoptric systems?}

There is ample literature that documents the benefits of natural light in
human-occupied spaces~\cite{Leslie03,ll01,pce00}, yet the thermal effects
can also be significant~\cite{bmbc13} (i.e., too much sunlight can
increase temperature to an unacceptable degree).

While many studies focus on the benefits of a homogeneous light levels
within a space~\cite{azaise13,bwkk15,gb16},
there is often differing need for light level based on user.
The increased control due to the independent positioning of the mirrors
can provide functional benefits for people that need more or less
light due to age of the occupant or their task. For example, someone using
a computer needs less light since the screen is a source of
artificial light while someone reading from physical media like
a book requires higher light levels. The light level required for
vision varies with the age of the occupants. As people get older,
they need higher light levels and increased luminous contrast.
Furthermore, everyone regardless of age have different qualities
of vision. Therefore, having a system that can accommodate these
varying conditions using daylight rather than artificial light is beneficial.

For thermal management, the literature on harvesting sunlight for space
heating is substantial (see, e.g.,~\cite{deW75,Hunt79,kbd76,Lunde80,smf08,wo06},
and we do not propose to contribute anything
new to that field.  We will simply leverage what are already well-understood
techniques. It is worth acknowledging that many sunlight-based heating systems
use catoptric systems (primarily for sun tracking). Our contributions here
will be to integrate a sunlight harvesting system into dual-purpose use, both
heating and illumination.

We will explore the use of Markov decision processes for multi-objective
control of catoptric systems.  We will start with a single objective,
matching a provided image map that represents a desired illumination pattern,
and then generalize to include the competing goal of thermal energy
harvesting.

For the purpose of designing an MDP, we will assume that a low-level
control mechanism exists that can position each individual mirror as
desired.  The MDP will model the higher-level management issues, e.g.,
which mirror should be pointed in which direction?
With a low-level controller in place, the state space, $\mathcal{X}$,
can encapsulate the set of mirrors pointed at each position in
the image map.  The set of actions, $\mathcal{A}$, will embody the movement
of one or more mirrors from their present position to a new position,
and the transition system, $T$, will encode the probabilistic variations
present in the available sunlight.  This allows the reward function, $R$,
to be a quantified match between the desired image map and the resulting
daylight that is directed to each position in the physical space.
There are any number of approaches to image comparison quantification
that can be used here~\cite{ds99,my09,wbo97}.

After an initial exploration of the MDP formulated as described above,
we will then proceed to expand the framework to include the additional
objective of harvesting heat.  We will quantify the benefits of each
objective using general utility functions~\cite{tggs10} in a manner previously
explored by our group. One candidate set of utility functions we will
investigate will be to prioritize daylighting performance, and only
allocate excess sunlight to the HVAC subsystem. A second alternative
will be to proportionally provide sunlight to each goal up to the point
that the illumination can no longer benefit, at which point all additionally
available sunlight goes into heating.

\FIXME{Add acceleration profile to MDP formulation.}

In each of the above MDPs under investigation, we will utilize the
value-optimal solution (guaranteed to exist within Markov decision process
theory); however, this solution typically must be computed off-line, since
it is, in general, exponentially expensive to compute.  In parallel, we
will explore the space of value-optimal solutions and seek to formulate
an inexpensive to compute heuristic that closely approximates the
true value-optimal solution.

\subsection{Question 2 -- How do we provide for the safety, reliability,
maintainability, and continued efficacy of these systems?}

The ability to control mirror position is only useful over time if
the additional considerations of safety, reliability, etc., are managed
properly for the system as a whole. To address these issues, we will
investigate the degree to which they can be handled within the context
of the low-level controller versus being dealt with within the high-level
complete system control.

We anticipate some of these considerations being incorporated as constraints
within the optimization process and others as additional goals that we
wish to maximize during the multi-objective optimization itself.  Let us
first consider those that will be addressed as constraints (e.g., safety).

\paragraph{Constraints.}
A commonly used safety constraint on a positioning subsystem is to
specify hard limits on the range of motion (e.g., of each mirror, individually).
This kind of limit can be imposed by the physical design of the pan-tilt
mechanics, simply by adding physical stops at the limit positions.

The more interesting constraint systems occur when the safety considerations
are no longer locally determinable, but are dependent upon context.
An example that is relevant for our catoptric surfaces is a limit on
the total light intensity that can be supported at various positions in the
field of view of the mirrors. When providing heat into the HVAC system,
we desire a high light concentration delivered to the heat transfer point.
When illuminating a physical space for human occupants, the above
levels of light concentration are not only undesirable, they are patently
unsafe.

These context sensitive constraints add two additional requirements.
First, the higher-level management system must be responsible for them.
In our MDP formulation, we much adjust the feasible state space, $\mathcal{X}$,
so that they are unreachable (or have sufficiently negative reward as to
amount to the same thing).
Second, we must also ensure that whatever choices are made by the optimization
system (whether it be via MDPs or some other approach), the actual
mirror positions are still constrained so as to not result in an
unsafe condition.  This will require feedback of some form on the actual
mirror positions and the ability to determine whether or not the actual
positions differ from the controlled positions.  This might be accomplished
either locally (e.g., via shaft encoders on the pan-tilt mechanism) or
globally (e.g., via a visual monitoring system and appropriate image analysis
software).

\paragraph{Goals.}
Many of the additional considerations are more appropriately addressed as
additional (potentially competing) goals, adding to the desires of
daylighting and heating.  An example here would be the
impact on reliability of the system when individual mirrors are moved.  As
with all electro-mechanical systems, the pan-tilt mechanism has a limited
lifetime, which can be significantly impacted by its usage duty cycle
(i.e., the more it is moved, the sooner it will fail).

This is precisely the set of circumstances in which Markov decision
processes excel. For the initial goals (of question~1), is is quite
likely that there are multiple optimal solutions (e.g., simply exchange
to pointing of two mirrors and the goals are unchanged).  However,
once we add in considerations of reliability, which is impacted by
frequency of use, we now have a much richer search space,
and a natural fit for probabilitic reward.
The MDP-based optimization approach is particularly will suited for
this type of problem, and can find optimal solutions that incorporate
minimal movement of mirrors between configurations, thereby diminishing
the maintainence costs during the normal use of the system.

The research task here is to develop, experiment with, and evaluate
Markov decision processes that simultaneously deliver daylight where it
is most beneficial, maintain the safety of the overall space,
and provide for the reliability, maintainability, and efficacy of the
catoptric surface itself.

\subsection{Question 3 -- Can we design abstractions that encapsulate
subsystems for effective reuse?}

Layered \emph{system architectures}\footnote{We use the term
\emph{system architecture}
to denote computer hardware/software architecture, distinguishing it from
architecture for the built world around us.}
have proven effective in providing standardized interfaces to
support portability
and reuse of hardware and software,
while allowing new innovations and abstractions
to enrich system capabilities.
Even when systems don't adhere strictly to standards
such as the OSI~\cite{osi} networking model or the POSIX~\cite{posix} operating
systems interface model, system designers, developers,
and users still benefit
from the structure those models provide.

This occurs largely because standardized interfaces establish clear
boundaries of
responsibility on which other layers may rely, while allowing an essential
``permission to tinker'' with various implementations between those
boundaries. For cyber-physical systems, whose semantics include timing
and physical properties not considered in previous cyber-only system
architectures, there is significantly less experience with what interfaces,
abstractions, and even broad system layers are truly common (and so perhaps
could be standardized) versus which other aspects are more likely to
diverge between systems, and so should be free to do so~\cite{cag18}.

In this research, we will investigate the viability and utility of
two candidate abstractions: direct mirror control and MDP control.

