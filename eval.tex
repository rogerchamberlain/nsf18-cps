\section{Evaluation/Experimentation Plan}
\label{sec:eval}

We will organize our experimentation and evaluation activities around
three prototype catoptric surfaces.  The first is the system currently
being installed in the south window of Steinberg Hall's atrium (described
in Section~\ref{sec:background}).
The second will be designed for and installed at VelociData, Inc.,
a startup firm in
St.~Louis that is located in the recently announced \emph{39~North}
innovation district.
The third will be designed for an installed at BECS Technology, Inc.,
a local manufacturer of electronic control systems that has recently
relocated to a newly redeveloped 42,000~sq.~ft. facility. 

\subsection{Steinberg Hall, Washington University in St. Louis}

Steinberg Hall is situated on the Danforth Campus of Washington University
in St.~Louis. It is one of the buildings that houses the College
of Architecture, and is within easy walking distance of the Dept. of
Computer Science and Engineering.

The catroptic surface that is being installed at the south end of the
atrium will be complete by the start date of the proposed research project.
We will use it for a number of purposes:
\begin{enumerate}

\item Development and calibration of quantitative daylight delivery models.
We will evaluate the effectiveness of our current ray-tracing software
system in assessing the impact of different configurations of the surface
(i.e., various mirror positions).  Empirical evaluation will use a number
of light meters distributed within the space. The data collected will be
compared to our predictions as well as the analytical models
of both Bueno et al.~\cite{bwkk15} and Galatioto and Beccali~\cite{gb16}.

\item Practical aspects.
We will learn several practical things from this installation, including
the positioning precision achievable with our current physical design, the
viability of operating the mirror positioning motors open loop (the pan-tilt
is stepper motor driven and the current system does not incorporate
shaft encoders or other positioning feedback), etc.

\item Investigation of the ability to provide positioning feedback via
image analysis.  We will install a camera with the full surface in its
field of view and assess the viability of using image analysis techniques
to discern the orientation of each mirror.  An imporant component of this
will be the degree of precision that is achievable.

\end{enumerate}

\subsection{Velocidata, Inc., 39 North innovation district}

10425 Old Olive Stree Rd., St.~Louis, MO.

\FIXME{VelociData is located in the recently announced \emph{39 North}
innovation district, which has the Danforth
Plant Science Center, Monsanto, Bio Research \& Development Growth Park,
and Heliz Center Biotech Incubator as anchors.}

\subsection{BECS Technology, Inc., St. Louis County}

10818 Midwest Industrial Dr., St.~Louis, MO.

\FIXME{BECS Technology is a small manufacturer of electronic control
systems in a number of markets (e.g., agriculture, aquatics, refrigeration).
We will have access to the HVAC system in their building.}
