\section{Results from Prior NSF Support}
\label{sec:prior}

\noindent
{\large\bf CSR: Small: Concurrent Accelerated Data Integration}
{\bf (CNS-1527510,
PI: R. Chamberlain)}, 
10/2015--9/2019, \$519,275.  

\textbf{Intellectual Merit} -- This project investigates the
accelerated execution of data integration workflows, which
increasingly are bottlenecks in data science. Execution platforms
being targeted include both graphics engines and FPGAs.  Publications
resulting from this work include~\cite{dibs,c17,mgc16,js16}.

\textbf{Broader Impacts} -- This research project has supported 3
graduate students and 4 REU students.  The applications investigated
come from the fields of computational biology, astrophysics, and the
Internet of Things, further expanding the scope of the students'
experience.  A benchmark suite of these workflows has been released
as a community resource~\cite{dibsv1}.

\textbf{Evidence of Research Products and their Availability} --

\noindent
{\large\bf CPS: Medium: Collaborative: CyberMech, a Novel Run-Time Substrate for 
Cyber-Mechanical Systems}
{\bf (CNS-1136073 and CNS-1136075,
Washington University: PI C. Gill, co-PIs Kunal Agrawal and Chenyang Lu; Purdue University: PI Arun Prakash, co-PI Shirley Dyke)}, 9/2011-8/2016, \$1,800,000 total.  

This research project developed novel foundations for parallel real-time computing, and used them to demonstrate the first ever real-time hybrid simulation involving a thousand-degree-of-freedom structure.

\textbf{Intellectual Merit} -- Results of this research include new methods for parallel real-time execution of control and simulation computations, new parallel real-time scheduling techniques and analyses, and characterization and exploitation of trade-offs involving both high computational demand and stringent timing constraints.

\textbf{Broader Impacts} -- This multi-university project involved 7 PhD, 3 masters, and 7 undergraduate students, and 2 visiting scholars in highly multi-disciplinary research collaborations.  Results of this research appeared in 10 publications at top-tier conferences and journals.

\textbf{Evidence of Research Products and their Availability} -- Data, experiment configurations, platform software, and simulation source-code have been published on-line at Washington University and Purdue University.

\begin{comment}
\vspace{0.1in}
\noindent
{\large\bf Very Energetic Radiation Imaging Telescope Array (VERITAS) Project}
{\bf (J. Buckley)}

This work was covered by a combination of a DOE research grant as well
as project funds in the form of DOE and NSF/Physics subcontracts
through the Smithsonian Astrophysical Observatory project office.

\textbf{Intellectual Merit} -- Significant products of the
work include the VERITAS FADC electronics and a number of scientific
results summarized in
\cite{2016MNRAS.461..202A,2016arXiv160901692A,2016arXiv160806464A,2016arXiv160801569A,2016MNRAS.459.2550A,2016ApJ...821..129A,2016ApJ...819..156B,2016ApJ...818L..33A,2016ApJ...817L...7A,2015ApJ...815L..22A,2015PhRvL.115u1103C,2015PhRvD..91l9903A,2015A&A...578A..22A,2015A&A...576A.126A,2015PDU.....7...16B,2015ApJ...800...61A,2015ApJ...799....7A,2014ApJ...797...89A,2014ApJ...795L...3A,2014ApJ...790..149A,2014ApJ...788..158A,2014ApJ...788...78A,2014ApJ...787..166A,2014ApJ...783...16A,2014ApJ...782...13A,2014APh....54....1A,2014arXiv1401.6085F,2014ApJ...780..168A,2013ApJ...779...92A,2013arXiv1310.8621A,2013arXiv1310.7040B,2013arXiv1310.5662B,2013ApJ...776...69A,2013arXiv1308.6173V,2013arXiv1307.4962P,2013arXiv1307.2807D,2013ApJ...770...93A,2013arXiv1305.0302W,2013arXiv1304.6367S,2013APh....43....3A,2013ApJ...764...38A,2013ApJ...762...92A}.
A founding member of VERITAS, Dr.\ Buckley was responsible for the design
and construction of the 2000-channel 500 Msps VERITAS FADC system.
He also played a leading role in establishing the scientific
programs for dark matter, multiwavelength studies of active galaxies
and observations of supernova remnants. Scientific highlights of
VERITAS include (1) DM limits on dwarf
galaxies\cite{2017PhRvD..95h2001A,2015PhRvD..91l9903A,2012PhRvD..85f2001A},
(2) resolved images and spectra from supernova remnants (SNR) leading
to direct evidence for the origin of hadronic cosmic rays in SNR
\cite{2017ApJ...836...23A,2013ApJ...764...38A,2013ApJ...770...93A,2010ApJ...719L..69A,2009ApJ...698L.133A},
(3) discovery of periodic emission from the Crab pulsar up to $>$100
GeV challenging current models of pulsar magnetospheres
\cite{2011Sci...334...69V}), (4) measurements of spectral variability
of active galaxies (AGNs) with multiwavelength data providing new
constraints on conditions near the central supermassive black hole
(e.g.,
\cite{2009ApJ...707..612A,2008ApJ...684L..73A,2011ApJ...726...43A,2009ApJ...691L..13D,2017ApJ...834....2A})
and (5) constraints on the extragalactic background light,
Lorentz-invariance violation and intergalactic magnetic fields
\cite{2012ApJ...750...94A,2010ApJ...708L.100A}.

\textbf{Broader Impacts} -- Information on VERITAS was disseminated to the
general public through YouTube videos, public site tours, and informational
displays at the Smithsonian Astrophysical Observatories. 

\textbf{Evidence of Research Products and their Availability} --

\end{comment}
